\documentclass{article}
\usepackage[a4paper, left=2cm, right=2cm, top=2cm, bottom=2cm,textwidth=17cm,textheight=23cm]{geometry}
\usepackage{xeCJK}
\usepackage{setspace}
\usepackage{fontspec}
\usepackage{fancyhdr}
\usepackage{hyperref}
\usepackage{graphicx}
\usepackage{xcolor}
\usepackage{amsmath, amssymb, amsthm}
\usepackage{booktabs}
\usepackage{tikz}
\setCJKmainfont{標楷體-繁}
\setmainfont{Times New Roman}
\pagestyle{fancy}
\fancyhf{}
\fancyfoot[C]{\thepage}
\fancyhead[L]{Explanation of Essay Questions}
\fancyhead[R]{\today}
\renewcommand{\headrulewidth}{0pt}
\renewcommand{\footrulewidth}{0pt}
\title{Explanation of Essay Questions}
\author{Tom T. Hsiao}
\date{\today}
\begin{document}
\setstretch{1.5}
\thispagestyle{fancy}
\begin{center}
\fontsize{16pt}{16pt}\selectfont Explanation of Essay Questions
\end{center}
\fontsize{14pt}{14pt}\selectfont
The purpose of essay questions is to assess students’ abilities in comprehension, analysis, application, and writing with respect to the given topic. Through the essay format, examiners can observe students’ understanding, assimilation, and application of course content, as well as their capacity to clearly express their arguments or viewpoints. This serves as a basis for evaluating their learning outcomes. The following provides an explanation of the essay grading criteria. \\
\\ \vspace{1em} 
\begin{tabular}{|c|p{15cm}|}
\hline
Criteria & Description \\
\hline
29 - 30 & The essay demonstrates a comprehensive understanding of the topic with more than three examples, with clear and logical organization. Arguments are well-supported with relevant examples. \\
\hline
26 - 28 & The essay shows a good understanding of the topic with at least two examples, organized in a logical manner. Arguments are generally supported with relevant examples. \\
\hline
22 - 25 & The essay provides a satisfactory understanding of the topic with at least an example. Arguments are somewhat supported with relevant examples. \\
\hline
18 - 21 & The essay demonstrates a basic understanding of the topic. \\
\hline
14 - 17 & The essay demonstrates a limited understanding of the topic. Arguments are somewhat supported without examples. \\
\hline
10 - 13 & The essay demonstrates a minimal understanding of the topic. Arguments are barely supported. \\
\hline
6 - 9 & The essay demonstrates a rare understanding of the topic. Arguments are unsupported. \\
\hline
1 - 5 & The essay demonstrates no understanding of the topic. Arguments are irrelevant. \\
\hline
0 & The essay is blank. \\
\hline
\end{tabular}
\end{document}
