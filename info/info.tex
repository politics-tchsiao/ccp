\documentclass{article}
\usepackage[a4paper, left=2cm, right=2cm, top=2cm, bottom=2cm,textwidth=17cm,textheight=23cm]{geometry}
\usepackage{xeCJK}
\usepackage{setspace}
\usepackage{fontspec}
\usepackage{fancyhdr}
\usepackage{hyperref}
\usepackage{xcolor}
\setCJKmainfont{標楷體-繁}
\setmainfont{Times New Roman}
\pagestyle{fancy}
\fancyhf{}
\fancyfoot[C]{\textsf{\thepage}}
\renewcommand{\headrulewidth}{0pt}
\renewcommand{\footrulewidth}{0pt}
\title{Politics}
\author{Tom T. Hsiao}
\date{}
\begin{document}
\setstretch{1.05}
\begin{center}
\fontsize{16pt}{16pt}\selectfont Politics \\
\end{center}
\fontsize{14pt}{14pt}\selectfont
\begin{flushleft}
Instructor Information
\end{flushleft}
\begin{itemize}
\item Instructor: Tzu-Chi Hsiao. \\
\item Office: . \\
\item Office hours: 13:20-13:50, Every Mondays. \\
\end{itemize}
Course Information \\
\begin{itemize}
\item Course: Politics. \\
\item Credit: 3. \\
\item Hours: 09:10-12:00, Every Mondays. \\
\item Prerequisite: High school civics. \\
\item Course description \\
In this course, we study a series of knowledge based on your high school's civics, and extend to subset of Political Science. \\ 
\item Course objectives \\
1. To understand the basic concepts of political science. \\
2. To understand the basic concepts of political philosophy. \\
3. To understand the basic concepts of public administration. \\
\end{itemize}
\begin{flushleft}
Grading \\
\end{flushleft}
\begin{itemize}
\item Midterm exam: 50\%. \\
\item Final exam: 50\%. \\
\end{itemize}
Textbook \\
\begin{itemize}
\item Politics, latest edition, written by Andew Heywood. \\
\end{itemize}
\end{document}
