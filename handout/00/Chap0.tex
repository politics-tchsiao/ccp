\documentclass{beamer}
\usepackage{handout}
\title{}
\author{}
\date{}
\begin{document}
\fontspec{Times New Roman}
\begin{frame}
\begin{center}
\Large{Politics} \\
\vspace{3em}
\normalsize{Instructor: Tzu-Chi Hsiao} \\
\vspace{3em}
\small{Department of Political Science} \\
\vspace{1em}
\small{National Taiwan University}
\end{center}
\end{frame}
\begin{frame}
\begin{center}
\Large{Chapter 0 Preliminary} \\
\vspace{3em}
\normalsize{Instructor: Tzu-Chi Hsiao} \\
\vspace{3em}
\small{Department of Political Science} \\
\vspace{1em}
\small{National Taiwan University} \\
\end{center}
\end{frame}
\begin{frame}{Course Information}
\begin{itemize}
\pause
\item Instructor: Tzu-Chi Hsiao.
\pause
\item Course: Politics.
\pause
\item Course ID: PS-1002.
\pause
\item Credit: 3.
\pause
\item Time: 09:10-12:00, Every Tuesday.
\pause
\item Office: R603, Social Science Building, NTU.
\pause
\item Office Hours: 08:10-09:00, Every Tuesday.
\end{itemize}
\end{frame}
\begin{frame}{Why taking this course}
\begin{itemize}
\pause
\item \textcolor{red}{Politics is happens everywhere and everyday.}
\pause
\item Based on high school's civics, we interest in the system, cause, and results.
\pause
\item We will given the fundamentals knowledge of politics.
\pause
\item Improve your skills on politics, we have assignments and examinations.
\end{itemize}
\end{frame}
\begin{frame}{Course Grading}
\begin{itemize}
\pause
\item Midterm: 50\%.
\vspace{1em}
\pause
\item Final: 50\%.
\end{itemize}
\end{frame}
\begin{frame}{Course Grading (Cont'd)}
\pause
Forms of examinations:
\pause
\begin{itemize}
    \item Explanation: 40\%, 8\% each.
    \pause
    \item Short essay: 60\%, 30\% each.
    \pause
    \item May have some True or False problems: 1\% each.
\end{itemize}
\pause
\begin{enumerate}
\item Examinations must be answer in Mandarin, excepts for foreigners attendance.
\pause
\item Explanation and short essay preparation are available in Self-Assessment.
\pause
\item Adjusting or not based on overall performance.
\end{enumerate}
\end{frame}
\begin{frame}{Course Textbook}
\pause
Main book: Politics, Newest\textsuperscript{th} edition by Andrew Heywood. \\
\pause
Reference books:
\begin{itemize}
\pause
\item Political Science: An introduction by Michael Roskin et al.
\end{itemize}
\end{frame}
\begin{frame}{Course Schedule}
\begin{center}
\begin{tabular}{|c|c|}
\hline
Week & Topic \\
\hline
1 & Course 00.\\
\hline
2 & Chapter 01, Chapter 02.\\
\hline
3 & Chapter 03.\\
\hline
4 & Chapter 04, Chapter 05.\\
\hline
5 & Chapter 06, Chapter 07.\\
\hline
6 & Chapter 08.\\
\hline
7 & Chapter 09, Chapter 10.\\
\hline
8 & \textcolor{red}{Midterm examination, 09:10-12:10.}\\
\hline
\end{tabular}
\end{center}
\end{frame}
\begin{frame}{Course Schedule (Cont'd)}
\begin{center}
\begin{tabular}{|c|c|}
\hline
Week & Topic \\
\hline
9 & Chapter 11.\\
\hline
10 & Chapter 12.\\
\hline
11 & Chapter 13, Chapter 14.\\
\hline
12 & Chapter 15, Chapter 16.\\
\hline
13 & Chapter 17, Chapter 18.\\
\hline
14 & Chapter 19.\\
\hline
15 & Chapter 20.\\
\hline
16 & \textcolor{red}{Final examination, 09:10-12:10.}\\
\hline
\end{tabular}
\end{center}
\end{frame}
\begin{frame}{Notice of examination}
    \begin{itemize}
    \pause
    \item Midterm examination: 09:10-12:10, Week 08.
    \pause
    \item Final examination: 09:10-12:10, Week 16.
    \pause
    \item Both examinations are held in R201, Social Science Building, NTU.
    \pause
    \item Exceed 09:40 cannot enter the examination room.
    \pause
    \item Examinees must bring student ID card to verify.
    \pause
    \item Early hand in opens after 10:40.
    \pause
    \item Using Mandarin to answer the problems.
    \pause
    \item \textcolor{red}{Cheating in is not allowed}.
    \end{itemize}
\end{frame}
\begin{frame}{Course Policy}
\begin{enumerate}
\pause
\item Without a specific reason, every caught cheating or plagiarism must dropped out from the course. \\
\end{enumerate}
\end{frame}
\begin{frame}{Any Questions}
\begin{center}
Feel free to contact me via \href{mailto:politics.tchsiao@gmail.com}{here}.
\end{center}
\end{frame}
\begin{frame}{共勉}
\normalsize{\textcolor{red}{學而時習之,不亦說乎?}} \\
\vspace{1em}
\normalsize{有朋自遠方來,不亦樂乎?} \\
\vspace{1em}
\normalsize{人不知而不慍,不亦君子乎?} \\
\vspace{1em}
\begin{flushright}
--《論語》
\end{flushright}
\end{frame}
\begin{frame}{}
\begin{center}
\Large{End of Chapter 0}
\end{center}
\end{frame}
\end{document}